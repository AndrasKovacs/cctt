% This is a template for submissions to HoTT 2023.
% The file was created on the basis of easychair.tex,v 3.5 2017/03/15
%
% Before submitting, rename the resulting pdf file to "yourname.pdf"
%
\documentclass[letterpaper]{easychair}
\usepackage{doc}
\usepackage{amsfonts}
\usepackage{bm}

\newcommand{\easychair}{\textsf{easychair}}
\newcommand{\miktex}{MiK{\TeX}}
\newcommand{\texniccenter}{{\TeX}nicCenter}
\newcommand{\makefile}{\texttt{Makefile}}
\newcommand{\latexeditor}{LEd}

\newcommand{\msf}[1]{\mathsf{#1}}
\renewcommand{\mit}[1]{\mathit{#1}}
\newcommand{\mbb}[1]{\mathbb{#1}}
\newcommand{\mbf}[1]{\mathbf{#1}}
\newcommand{\bs}[1]{\boldsymbol{#1}}
\newcommand{\Nat}{\mbb{N}}
\newcommand{\suc}{\msf{suc}}

\bibliographystyle{alphaurl}

%  \authorrunning{} has to be set for the shorter version of the authors' names;
% otherwise a warning will be rendered in the running heads. When processed by
% EasyChair, this command is mandatory: a document without \authorrunning
% will be rejected by EasyChair
\authorrunning{}

% \titlerunning{} has to be set to either the main title or its shorter
% version for the running heads. When processed by
% EasyChair, this command is mandatory: a document without \titlerunning
% will be rejected by EasyChair
\titlerunning{}

%
\title{Efficient Evaluation for Cubical Type Theories
% \thanks{Your acknowledgements.}
}

\author{
András Kovács
}

\institute{
  Eötvös Loránd University\\
  \email{kovacsandras@inf.elte.hu}
}

\begin{document}
\maketitle

Currently, cubical type theories are the only known systems which support
computational univalence. We can use computation in these systems to shortcut
some proofs, by appealing to definitional equality of sides of
equations. However, efficiency issues in existing implementations often preclude
such computational proofs, or it takes a large amount of effort to find
definitions which are feasible to compute. In this abstract we investigate the
efficiency of the ABCFHL \cite{abcfhl} Cartesian cubical type theory with
separate homogeneous composition ($\msf{hcom}$) and coercion ($\msf{coe}$),
although most of our findings transfer to other systems.

\subsubsection*{Cubical normalization-by-evaluation}

In variants of non-cubical Martin-Löf type theory, definitional equalities are
specified by reference to a substitution operation on terms. However, well-known
efficient implementations do not actually use term substitution. Instead,
\emph{normalization-by-evaluation} (NbE) is used, which corresponds to certain
\emph{environment machines} from a more operational point of view. In these
setups, there is a distinction between syntactic terms and semantic
values. Terms are viewed as immutable program code that supports evaluation into
the semantic domain but no other operations.

In contrast, in cubical type theories interval substitution is an essential
component of computation which seemingly cannot be removed from the
semantics. Most existing implementations use NbE for ordinary non-cubical
computation, but also include interval substitution as an operation that acts on
semantic values. Unfortunately, a naive combination of NbE and interval
substitution performs poorly, as it destroys the implicit sharing of work
and structure which underlies the efficiency of NbE in the first place. We
propose a restructured cubical NbE which handles interval substitution more
gracefully. The basic operations are the following.

\begin{enumerate}
\item \emph{Evaluation} maps from syntax to semantics like before, but it
  additionally takes as input an interval environment and a cofibration.
\item \emph{Interval substitution} acts on values, but it has trivial cost by itself;
  it only shallowly stores an explicit substitution.
\item \emph{Forcing} computes a value to weak head form by sufficiently computing
  previously stored delayed substitutions.
\end{enumerate}


On canonical values, forcing simply pushes substitutions further down, incurring
minimal cost. But on neutral values, since neutrals are not stable under
substitution, forcing has to potentially perform arbitrary computation. Here we
take a hint from the formal cubical NbE by Sterling and Angiuli
\cite{ctt-normalization}, by annotating neutral values with stability
information. This allows us to quickly determine whether a neutral value is
stable under a given substitution. When it is stable, forcing does not have to
look inside it.

It turns out that there is only a single computation rule in the ABCFHL theory
which can trigger interval substitution with significant cost: the coercion rule
for the $\msf{Glue}$ type former. In every other case, only a \emph{weakening}
substitution may be created, but all neutral values are stable under weakening,
so forcing by weakening always has a trivial cost.

\subsubsection*{Using canonicity in closed evaluation}

In non-cubical type theories, evaluation of closed terms can be more efficient
than that of open terms. For instance, when we evaluate an
$\msf{if\!-\!then\!-\!else}$ expression, we know that exactly one branch will be
taken. In open evaluation, the $\msf{Bool}$ scrutinee may be neutral, in which
case both branches may have to be evaluated.

In the cubical setting, \emph{systems of partial values} can be viewed as
branching structures which make case distinctions on cofibrations. Importantly,
there are computation rules which scrutinize \emph{all} components of a cubical
system. These are precisely the homogeneous composition rules ($\msf{hcom}$) for
strict inductive types. For example:
\[ \msf{hcom}^{r\to r'}_{\mbb{N}}\,[\psi \mapsto i.\,\suc\,t]\,(\suc\,b) =
\suc\,(\msf{hcom}^{r\to r'}_{\mbb{N}}\,[\psi \mapsto i.\,t]\,b) \]
When we only have interval variables and a cofibration in the context, we do not
have to compute every system component to check for $\suc$. In this case, which
we may call ``closed cubical'', we can use the canonicity property of the
theory. Here $\suc\,b$ in the $\msf{hcom}$ base implies that every system
component is headed by $\suc$ as well. Hence, we can use the following rule
instead:
\[ \msf{hcom}^{r\to r'}_{\mbb{N}}\,[\psi \mapsto i.\,t]\,(\suc\,b) =
   \suc\,(\msf{hcom}^{r\to r'}_{\mbb{N}}\,[\psi \mapsto i.\,\msf{pred}\,t]\,b) \]
Here, $\msf{pred}$ is a metatheoretic function which takes the predecessor of a
value which is already known to be definitionally $\suc$. The revised rule
assumes nothing about the shape of $t$ on the left hand side, so we can compute
$\msf{pred}$ lazily in the output. These lazy projections work analogously for
all non-higher inductive types. For higher-inductive types, $\msf{hcom}$ is a
canonical value, so there is no efficiency issue to begin with.

\subsubsection*{Summary}

\begin{itemize}
  \item Costly interval substitution can only arise from computing with $\msf{Glue}$ types.
  \item In closed cubical evaluation, no computation rule forces all components of a system.
\end{itemize}
We are optimistic that an implementation with these properties would yield significant
performance improvements. We are currently in the process of developing this system
and adapting existing benchmarks to it.

\bibliography{references}

\end{document}
